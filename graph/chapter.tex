\section{Grafy}

\subsection{Przepływy}
\cfile{dinic.cpp}
\cfile{mcmf.cpp}

\subsubsection{Przepływy z wymaganiami}
Szukamy przepływu $\leq F$ takiego, że $f_i \geq d_i$ dla każdej krawędzi.
Tworzymy nowe źródło $s'$ i ujście $t'$.
Następnie dodajemy krawędzie
\begin{compactitem}
  \item $(u_i, t', d_i)$, $(s', v_i, d_i)$, $(u_i, v_i, c_i - d_i)$ zamiast $(u_i, v_i, c_i, d_i)$ 
  \item $(t, s, F)$
\end{compactitem}
Przepływ spełnia wymagania jeżeli maksymalnie wypełnia wszystkie krawędzie $s'$.

\subsection{Grafy dwudzielne}
\cfile{matching.cpp}

\subsubsection{Rozszerzone twierdzenie K\"oniga}
W grafie dwudzielnym zachodzi
\begin{compactitem}
  \item $\mathrm{nk} = \mathrm{pw}$
  \item $\mathrm{nk} + \mathrm{pk} = n$
  \item $\mathrm{pw} + \mathrm{nw} = n$
\end{compactitem}
oraz
\begin{compactitem}
  \item pw to zbiór wierzchołków na brzegu min-cut
  \item nw to dopełnienie pw
  \item pk to nk z dodanymi pojedynczymi krawędziami każdego nieskojarzonego wierzchołka
\end{compactitem}

\subsection{Grafy skierowane}
\cfile{scc.cpp}
