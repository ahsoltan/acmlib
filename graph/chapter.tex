\section{Grafy}

\subsection{Przepływy}
\cfile{Dinic.h}
\cfile{MCMF.h}

\subsubsection{Przepływy z wymaganiami}
Szukamy przepływu $\leq F$ takiego, że $f_i \geq d_i$ dla każdej krawędzi.
Tworzymy nowe źródło $s'$ i ujście $t'$.
Następnie dodajemy krawędzie
\begin{compactitem}
  \item $(u_i, t', d_i)$, $(s', v_i, d_i)$, $(u_i, v_i, c_i - d_i)$ zamiast $(u_i, v_i, c_i, d_i)$ 
  \item $(t, s, F)$
\end{compactitem}
Przepływ spełnia wymagania jeżeli maksymalnie wypełnia wszystkie krawędzie $s'$.

\subsection{Grafy dwudzielne}
\cfile{Matching.h}

\subsubsection{Twierdzenie K\"oniga}
W grafie dwudzielnym zachodzi
\begin{compactitem}
  \item $\mathrm{nk} = \mathrm{pw}$
  \item $\mathrm{nk} + \mathrm{pk} = n$
  \item $\mathrm{pw} + \mathrm{nw} = n$
\end{compactitem}
oraz
\begin{compactitem}
  \item pw to zbiór wierzchołków na brzegu min-cut
  \item nw to dopełnienie pw
  \item pk to nk z dodanymi pojedynczymi krawędziami każdego nieskojarzonego wierzchołka
\end{compactitem}

\subsubsection{Twierdzenie Gale'a-Rysera}
Ciągi stopni $a_1 \geq \dots \geq a_n$ oraz $b_1, \dots, b_n$ opisują
prosty graf dwudzielny wtw gdy $\sum a_i = \sum b_i$ oraz dla każdego $1 \leq k \leq n$ zachodzi
\[
  \sum_{i=1}^{k} a_i \leq \sum_{i=1}^{n} \min(b_i, k).
\]

\subsection{Grafy skierowane}
\cfile{SCC.h}

\subsection{Grafy nieskierowane}
\subsubsection{Twierdzenie Erd\H{o}sa-Gallaia}
Ciąg stopni $d_1 \geq \dots \geq d_n$ opisuje prosty graf wtw gdy
$\sum d_i$ jest parzysta oraz dla każdego $1 \leq k \leq n$ zachodzi
\[
  \sum_{i=1}^{k} d_i \leq k(k - 1) + \sum_{i=k + 1}^{n} \min(d_i, k).
\]
